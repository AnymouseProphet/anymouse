\documentclass[letterpaper, fontsize=14pt]{scrartcl}
\setlength{\parskip}{\baselineskip}
\setlength{\parindent}{0cm}
\usepackage{iftex}
\ifpdftex
  \usepackage[utf8]{inputenc}
  \renewcommand{\rmdefault}{qtm}
  \renewcommand{\sfdefault}{qhv}
\else
  \usepackage{fontspec}
  \setmainfont{Tex Gyre Termes}
  \setsansfont{Tex Gyre Heros}
\fi

\usepackage[fontspec,lgrfont]{amplgrgreek}
%
% The fontspec is meaningless w/ pdfLaTeX. With LuaLaTeX, fontspec=no causes
%  fontspec not to be used to load the font. Also, if the font can't be found
%  then it is suppose to use the 8-bit version but right now that is sort of
%  broken. \IfFileExists{} doesn't seem work with font files.
%
% Also with LuaLaTeX have to already load fontspec before calling this package.
%
% The lgrfont is for specifying a font that has an 8-bit LGR encoded family
%  within the TeX TDS. The package looks for a file named lgr<name>.fd so
%  for example if you wanted GFS Bodoni use lgrfont=bodoni
% If the lgr fd file isn't found, it uses artemisia rather than failing.






\begin{document}
\raggedright

Current Default Font Encoding: \encodingdefault

This is a test document for a Greek class I am working on. The purpose of the class is to allow non-Greek documents to include Greek strings in a portable way that works in both pdf\LaTeX{} and in Lua\LaTeX{}.

The user \emph{should} be able to input Greek using either UTF-8 or by using the 8-bit LGR encoding.

With Lua\LaTeX{} fontspec should be used to load the font when fontspec is already loaded \emph{before} calling my package and my package can find the Unicode version of the requested font. The user should also be able to disable the use of fontspec to load the font.

Right now things work as expected with pdf\LaTeX{} but there are two bugs when using Lua\LaTeX{}.

I know people think fontspec makes working with fonts easier, and maybe that is true with RTL scripts or with fonts that do not fit in 256 slots (half of which are always ASCII and control characters) it seems to me pdf\LaTeX{} is still easier when it comes to using fonts.

\section{Testing Greek Output with UTF8 Input}

This is some \textbf{English} content.

\amptextgreek{Αυτό είναι \textbf{ελληνικό} περιεχόμενο.}

\rule{4cm}{0.4pt}

Unfortunately the above test only works with pdf\LaTeX{} or with Lua\LaTeX{} when fontspec is used. When fontspec is not used, that test fails in Lua\LaTeX{}. I believe that is a bug with Lua\LaTeX{} not properly handling\linebreak
\texttt{\textbackslash fontencoding\string{LGR\string}} with utf8 input.

I tried both

\texttt{\textbackslash usepackage[utf8]\string{inputenc\string}} and\linebreak
\texttt{\textbackslash usepackage[utf8]\string{luainputenc\string}} and neither solved the issue.

When fontspec is not used, resulting in 8-bit fonts as is always the case with pdf\LaTeX{}, pdf\LaTeX{} passes the test (as long as\linebreak
\texttt{\textbackslash usepackage[utf8]\string{inputenc\string}} has been called) but Lua\LaTeX{} fails.

Lua\LaTeX{} is not properly converting the UTF-8 input into the 8-bit LGR encoding needed for the \texttt{substitutefont} package to do its thing.

\section{Testing Greek Output with LGR Input}

This \textbf{content} is not \textit{typed} in Unicode.

\amptextgreek{Aut\'o to \textbf{perieq\'omeno} den \'eqei \textit{plhktrologhje\'i} sto} Unicode

\rule{4cm}{0.4pt}

The above is \emph{expected} to fail in Lua\LaTeX{} with a fontspec loaded font because I haven't yet found a way to translate from 8-bit LGR to UTF-8 Unicode that the Unicode font needs. Not a Lua\LaTeX{} bug, but a feature I specifically want to add to my package so that it truly is portable between pdfLaTeX{} and Lua\LaTeX{}.

Without a fontspec loaded font, it works in both pdf\LaTeX{} and Lua\LaTeX{} as it is 8-bit input being rendered with an 8-bit font and (whether or not fontspec is being used for other fonts) the \texttt{substitutefont} package can do its thing.

\end{document}
